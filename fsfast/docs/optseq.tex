\documentclass[10pt]{article}
\usepackage{amsmath}

%%%%%%%%%% set margins %%%%%%%%%%%%%
\addtolength{\textwidth}{1in}
\addtolength{\oddsidemargin}{-0.5in}
\addtolength{\textheight}{.75in}
\addtolength{\topmargin}{-.50in}

%%%%%%%%%%%%%%%%%%%%%%%%%%%%%%%%%%%%%%%%%%%%%%%%%%%%%%%%%%%%%%%%%
%%%%%%%%%%%%%%%%%%%%%%% begin document %%%%%%%%%%%%%%%%%%%%%%%%%%
%%%%%%%%%%%%%%%%%%%%%%%%%%%%%%%%%%%%%%%%%%%%%%%%%%%%%%%%%%%%%%%%%
\begin{document}

\begin{Large}
\noindent {\bf optseq} \\
\end{Large}

\noindent 
\begin{verbatim}
Comments or questions: analysis-bugs@nmr.mgh.harvard.edu
\end{verbatim}

\section{Introduction}

When stimuli are presented so closely together that their hemodynamic
responses overlap, it becomes necessary to ``jitter'' the stimulus
onsets in order to compute the hemodynamic response without assuming a
shape. This jitter is realized by inserting random amounts of {\em
null} events between task events; ie, the stimulus onset asynchrony
(SOA) is randomized. NOTE: the null event (also known as ``fixation'')
is not to be considered an event. When designing such an experiment,
one must determine the order of stimulus event type as well as how
much null to put between each. This combination is referred to as the
{\em schedule}. Even for simple designs, there is a huge number of
possibilities, and not all of them are the same. 

{\bf optseq} is a program for optimizing schedules given the various
parameters of the experimental design and analysis, including TR,
temporal estimation resolution (TER), number of time points (or scans)
to be collected, the time window over which the hemodynamic response
will be estimated, and the number of presentations per run for each
condition. Note that optseq will not determine the optimum average
length of the null event (ie, average SOA). 

The optimization criterion is based on the following forward model:
\begin{equation}
y = M X h + n,
\label{fmrix.eqn}
\end{equation}
where $y$ is the measured BOLD signal, $M$ is a subsampling matrix
(may be the identity), $X$ is the stimulus convolution matrix
dependent only upon the stimulus sequence, $h$ is the hemodynamic
response, and $n$ is noise.  If the noise is white, then the expected
RMS error is $e_{RMS} = \sigma_n^2 trace((X^t X)^{-1})$.  Note that
under these assumptions the error is dependent only on the stimulus
sequence $X$ and noise variance $\sigma_n^2$.  The noise variance is
beyond the experimenter's control; however, the RMS error can still be
manipulated by chaning the stimulus schedule $X$.  We define the {\em
efficiency} as $E = \frac{1}{trace((X^t X)^{-1})}$. Optseq
searches candidate sequences for the ones that maximize the efficiency
(and so minimize the expected RMS error).\\

The output sequences are stored in the {\em paradigm file format}.
This is a text file with (at least) two columns.  The first column
indicates the time at which a stimulus is to be presented, where time
zero is implicitly assumed to be the onset of the first data
acquisition. The second column indicates the identification number of
the type of stimulus presented.  Each event type has its own unique
number which is assigned by optseq, with identification number 0 being
reserved for the null condition and increasing contiguously for other
conditions.  For {\bf optseq} output, a third column is added
indicating the duration that the stimulus is to be presented.  This is
redundant with the first column but can be convenient for converting
to a file used by a stimulus presentation program.\\

\section{Usage}
Typing optseq at the command line without any options will give the
following message:\\ 

\begin{small}
\begin{verbatim}
USAGE: optseq [-options] -ntp ntp -npercond n1 <n2 ...> -o outprefix
   -npercond n1 <n2 ...> : number of stimuli for each non-null condition in a session
   -tpercond t1 <t2 ...> : duration of each stimulus type 
   -o outprefix          : prefix of output paradigm files 
   -ntp  ntp             : Number of time points to be collected for each run
Options:
   -label l1 <l2 ...> : label for each non-null condition
   -nsessions <int> : number of sessions (1)
   -nruns <int>     : number of runs in each session (1)
   -TR    <float>   : TR to use (2 sec)
   -TER   <float>   : Temporal resolution of the estimate (TR)
   -Tres  <float>   : Temporal resolution of the stimulus presentation (TER)
   -tprescan <float>   : time before first scan to present first stim (0)
   -timewindow <float> : Time Window of estimate (20 sec)
   -prestim   <float>  : part of timewindow to use for prestim baseline (0)
   -nsearch <int>   : number of sequences over which to search (100)
   -tsearch <float> : number of hours over which to search 
   -seed    <int> : seed for random number generator (auto)
   -mail  user    : mail user when finished
   -monly mfile   : generate matlab file only
\end{verbatim}
\end{small}

\section{Command-line Arguments}

\noindent
{\bf -npercond n1 <n2 ...> }: this is the number of stimuli per
condition per session where $n1$ is the number of times that condition
type 1 will be presented; $n2$ is the number of times type
2 is presented; etc. List as many as there are conditions. There must
be at least one condition. Do not list the null condition.  The
identification number of each condition (as found in the paradigm
file) corresponds to the sequence in which it is listed after the
$npercond$ flag.  Identification number 0 always corresponds to the
null condition.\\

\noindent
{\bf -tpercond t1 <t2 ...> }: this is the amount of time (in seconds)
that will be allocated for each event presentation.  $t1$ is the time
for condition 1, $t2$ is the time for condition 2, etc.  Note that
this allows for conditions to be of different durations. {\em Durations
that are not integer multiples of the TER will be rounded up to the
next TER}.

\noindent
{\bf -o outprefix}: this is the prefix used to construct the name of
the paradigm files.  The paradigm file for a run will be {\em
outprefix-sXXX-rYYY.dat} where $YYY$ is the three-digit, zero-padded session
number, and $XXX$ is the three-digit, zero-padded run
number.\\ 

\noindent
{\bf -Ntp ntp}: number of scans, aquisitions, or time-points to be
collected for each run.\\

\noindent
{\bf -label l1 <l2 ...>}: list of labels for each non-null condition.
This label will be appended to each line of the paradigm file where
the condition is presented.  It is for ease of reference only.\\

\noindent
{\bf -nsessions }: number of sessions.  Each session has a number of
runs specified by the $-nruns$ flag.  A paradigm file will be
generated for each run, but the optimization procedure will be applied
to all the runs in a session.  Increasing the number of sessions does
not significantly increase the amount of time it takes the program to
run.\\ 

\noindent
{\bf -nruns }: number of runs per session.\\

\noindent
{\bf -TR TR}: temporal resolution of the fMRI scans (ie, the time
between volume aquisitions) in seconds.\\

\noindent
{\bf -TER TER}: temporal resolution of the estimate of the hemodynamic
response in seconds. Default is to set the TER to the TR.  If sub-TR
estimation is desired, the TER must be an integer divisor of the TR.\\

\noindent
{\bf -Tres Tres}: temporal resolution (in seconds) of the stimulus
presentation time.  This sets the resolution of the ``clock'' used to
present the stimuli.  The presentation time is the first column in the
paradigm file, and each time entry will be an integer multiple of
$Tres$. By default, $Tres$ is set to the TER, and it would be unusual
to do otherwise. \\

\noindent
{\bf -tprescan}: time (in seconds) before the first acquistion of a
run to begin presenting stimuli. \\

\noindent
{\bf -timewindow <float>}: time window in seconds over which the
hemodynamic response will be estimated.  This should be equal to the
maximum stimulus duration plus the expected hemodynamic recovery time
plus any prestimulus estimation time (see $-prestim$). Default is 20
sec.\\

\noindent
{\bf -prestim }: specify a portion of {\em timewindow} to
estimate the hemodynamic response {\em before} stimulus onset.\\

\noindent
{\bf -nsearch, -tsearch }: The number of possible legal sequences is
enormous and cannot be exhaustively searched.  Therefore, optseq
performs a random search of legal sequences and keeps the ones that
are best.  These two options determine how long optseq will search
before returning an answer.  $-nsearch$ allows the user to specify the
number of sequences to sample. Alternatively, $-tsearch$ specifies the
number of hours that optseq should run before returning an answer.  It
would not be unusual to let optseq run for 12 hours, and several days
would not be out of the question.\\

\noindent
{\bf -seed }: specifies the seed for the random number generator.  By
default, optseq will automatically choose a seed based on the
computer's system clock. Setting the seed to -1 explicitly forces
optseq to choose a seed.\\

\noindent
{\bf -mail user}: send mail to user when the program finishes.  Handy
for those long search times.\\

\noindent
{\bf -monly mfile}: only generate the matlab file which would accomplish the
analysis but do not actually execute it.  This is mainly good for
debugging purposes.\\

\section{Selecting Parameters}

Optseq requires that the user supply the following parameters: 
\begin{enumerate}
\item Number of presentations per condition/event-type ($N_{pc}$)
\item TR/TER
\item Duration of each condition/event-type presentation ($T_{pc}$)
\item Total number of time points ($N_{tp}$), Average Stimulus Onset
Asynchrony (ASOA).
\end{enumerate}

The selection of these parameters depends upon what the experimenter
is trying to accomplish as well as the realities of analyzing the
data.

The number of times each event type is presented will affect how
reliably the response to that event type can be distinguished from
fixation and other event types. The minimum number of presentations
needed to achieve a given level of significance is dependent upon the
effect size, which is usually not known in advance (and is often the
object of the experiment). However, to see a main effect (eg,
responses in visual cortex), one generally needs about 20
presentations. To see a subtle effect in a cognitive area, one may need
as many as 40 presentations. In general, each event type should be
presented the same number of times. Note that -npercond expects the
total number of presentations over the entire session.

The TR needs only be fast/short enough to capture changes in the
hemodynamic response. A TR of 2 seconds seems to work well for this.
There is an inverse relationship between the TR and the number of
slices that can be acquired, so, while a very fast/short TR is
generally better for signal processing, it will reduce the number of
slices and so the amount of brain coverage. If the TR must be extended
beyond 2 seconds, then one can still reconstruct the hemodynamic
response at a resolution finer than the TR by specifying the TER
(temporal estimation resolution). The TER must be an integer divisor
of the TR. If the TER equals the TR, then events will always start at
the beginning of the TR.  TR. If the TER equals half the TR, then
events will always start at either the beginning of the TR or midway
through the TR. A TER less than the TR will allow the hemodynamic
response to be estimated at a finer resolution; however, it can
significantly reduce the power of subsequent statistical tests (if one
does not assume a shape to the hemodynamic response).

The duration of each event type is usually up to the
experimenter. However, optseq will work best if the duration is an
integer multiple of the TER. The duration of each event type will, of
course, affect the total amount of scanning time.

Choosing the number of time points.  Once the duration of each event
type and the number of times each event type will be presented are
determined, one can compute the total amount of time that will be
needed to present all the task-related stimuli:

\begin{equation}
T_{stim,tot} = \sum_{i=1}^{N_c} N_{pc,i}T_{pc,i},
\end{equation}
where $N_c$ is the number of non-null event types. As a minimum, then,
the number of time points must be $\frac{T_{stim,tot}}{TR}$. However,
this does not account for presentations of the null stimulus needed to
optimize the schedule. As a rule of thumb, the total time
allocated to presentation of the null stimulus should equal to the
average time given to the task conditions. This is equal to:
\begin{equation}
T_{null,tot} = \frac { T_{stim,tot} }
                     { \sum_{i}{N_c}N_{pc,i} }
\end{equation}
This makes the total time needed to present all stimuli (task and
null):
\begin{equation}
T_{tot} = T_{null,tot} +  T_{stim,tot} 
\end{equation}
The number of time points needed then will be $\frac{T_{tot}}{TR}$.
Note: Optseq forces the null stimulus duration to be an integer
multiple of the TER.


\section{Example}

Consider an experiment with 2 sessions and 3 runs per session.  There
will be 120 acqusitions per run with a TR of 2 seconds.  There are
three non-null event types (or conditions): ``normal'', ``anomolous'',
and ``nonsense''.  Over all 3 runs, there will be 60 presentations of
the normal stimulus type, and each presentation will last 2 seconds.
There will be 65 presentations of the anomalous stimulus type, and
each presentation will last 1 second.  There will be 44 presentations
of the nonsense stimulus type, and each presentation will last 3
seconds.  The data will be analyzed with a temporal estimation
resolution 1 second over a window of 20 seconds, 4 seconds of which
will be used to acquire a prestimulus baseline.  Paradigm files for
this design can be generated by running optseq with the following options:

\begin{small}
\begin{verbatim}
optseq  \
  -nsessions 2 -nruns 3 -TR 2 -ntp 120   \
  -npercond 60 65 44   \
  -label normal anomalous nonsense   \
  -tpercond 2 1 3   \
  -timewindow 20  \
  -tprestim 4   \
  -TER 1   \
  -o par   \
  -tsearch .05 
\end{verbatim}
\end{small}

This instructs optseq to run for .05 hours (only 3 minutes).  It will
produce six paradigm files: par-s001-r001.dat, par-s001-r002.dat,
par-s001-r003.dat, par-s002-r001.dat, par-s002-r002.dat, and
par-s002-r003.dat. It will also produce a log file as shown below.

\begin{small}
\begin{verbatim}
par.log
Wed Jan  5 20:15:31 EST 2000
Nsessions:  2
Nruns:      3
Ntp:        120
nPerCond:   60 65 44
tPerCond:   2 1 3
TR:         2 seconds
TER:        1 seconds
Tres:       1 seconds
TimeWindow  20 seconds
TPreStim    4 seconds
TPreScan    0 seconds
nsearch:    0
tsearch:    .05
seed = 2.056159e+05
Event Types:
 1 60 2.000 normal   
 2 65 1.000 anomalous
 3 44 3.000 nonsense 
tsearched = 180.019
nsearched = 1796
Session   1  Efficiency = 0.642871
Session   2  Efficiency = 0.640874
Wed Jan  5 20:18:33 EST 2000
\end{verbatim}
\end{small}

The log file shows the arguments with which optseq was invoked as well
as a summary of results of the search.  The search started on $Wed Jan
5 20:15:31 EST 2000$ and ended on $Wed Jan 5 20:18:33 EST 2000$.
Optseq automatically chose a seed of $2.056159e+05$.  The time of the
search was 180 seconds (ie, .05 hours) during which it sampled 1769
sequences of which it kept 2.  The best sequence (Session 1) had an
efficiency of 0.642871. The next best sequence (Session 2) had an
efficiency of 0.640874.

The first 10 lines of par-s001-r001.dat are:

\begin{small}
\begin{verbatim}
   0.000    3     3.000     nonsense
   3.000    1     2.000     normal
   5.000    1     2.000     normal
   7.000    0     1.000     null
   8.000    2     1.000     anomalous
   9.000    0     1.000     null
  10.000    3     3.000     nonsense
  13.000    0     1.000     null
  14.000    1     2.000     normal
\end{verbatim}
\end{small}

The first column is the time at which the event begins with respect to
the first image to be acquired.  The second column is the ID of the
event.  The third column is the duration of the event, and the fourth
column is the label of the event.

\end{document}
