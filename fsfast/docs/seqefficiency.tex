\documentclass[10pt]{article}
\usepackage{amsmath}

%%%%%%%%%% set margins %%%%%%%%%%%%%
\addtolength{\textwidth}{1in}
\addtolength{\oddsidemargin}{-0.5in}
\addtolength{\textheight}{.75in}
\addtolength{\topmargin}{-.50in}

%%%%%%%%%%%%%%%%%%%%%%%%%%%%%%%%%%%%%%%%%%%%%%%%%%%%%%%%%%%%%%%%%
%%%%%%%%%%%%%%%%%%%%%%% begin document %%%%%%%%%%%%%%%%%%%%%%%%%%
%%%%%%%%%%%%%%%%%%%%%%%%%%%%%%%%%%%%%%%%%%%%%%%%%%%%%%%%%%%%%%%%%
\begin{document}

\begin{Large}
\noindent {\bf seqefficiency} \\
\end{Large}

\noindent 
\begin{verbatim}
Comments or questions: analysis-bugs@nmr.mgh.harvard.edu
\end{verbatim}

\section{Introduction}
{\bf seqefficiency} is a program measuring the efficiency of an
event-related stimulus sequence. Requires matlab 5.2 or higher.\\

\section{Usage}
Typing seqefficiency at the command line without any options will give the
following message:\\ 

\begin{small}
\begin{verbatim}
USAGE: seqefficiency -p parfile -Ntp ntp [options] 
   -p parfile  : name of the paradigm file
   -Ntp ntp    : Number of timepoints collected
Options:
   -TR    <float> : TR to use (2 sec)
   -TER   <float> : Temporal resolution of the estimate (TR)
   -timewindow <float> : Time Window to use (20 sec)
   -monly         : generate matlab file only
\end{verbatim}
\end{small}

\section{Command-line Arguments}

\noindent
{\bf -p parfile }: name of the paradigm file in which the sequence is
located. \\

\noindent
{\bf -Ntp ntp}: number of scans or time-points to be collected for
each run.\\

\noindent
{\bf -TR TR}: temporal resolution of the fMRI scans (ie, the time
between scans) in seconds.\\

\noindent
{\bf -TER TER}: temporal resolution of the estimate of the hemodynamic
response in seconds. The stimuli will be presented at increments of
the TER.  Default is to set the TER to the TR.\\

\noindent
{\bf -timewindow <float>}:  time window in seconds over which the
hemodynamic response will be estimated.  Default is 20 sec.\\

\noindent
{\bf -monly}: only generate the matlab file which would accomplish the
analysis but do not actually execute it.  This is mainly good for
debugging purposes.\\

\section{Output}

{\bf seqefficiency} prints the efficiency measure to the standard output.\\

\end{document}
