\documentclass[10pt]{article}
\usepackage{amsmath}
%\usepackage{draftcopy}

%%%%%%%%%% set margins %%%%%%%%%%%%%
\addtolength{\textwidth}{1in}
\addtolength{\oddsidemargin}{-0.5in}
\addtolength{\textheight}{.75in}
\addtolength{\topmargin}{-.50in}

%%%%%%%%%%%%%%%%%%%%%%%%%%%%%%%%%%%%%%%%%%%%%%%%%%%%%%%%%%%%%%%%%
%%%%%%%%%%%%%%%%%%%%%%% begin document %%%%%%%%%%%%%%%%%%%%%%%%%%
%%%%%%%%%%%%%%%%%%%%%%%%%%%%%%%%%%%%%%%%%%%%%%%%%%%%%%%%%%%%%%%%%
\begin{document}

\begin{Large}
\noindent {\bf mkcontrast} \\
\end{Large}

\noindent 
\begin{verbatim}
Comments or questions: analysis-bugs@nmr.mgh.harvard.edu
\end{verbatim}

\section{Introduction}

{\bf mkcontrast} is a program for creating a contrast matrix. The
output of mkcontast can be used to specify contrasts for {\em
stxgrinder} and for {\em isxavg-re}. The contrast matrix $R$ is used
to test the hypothesis
\begin{equation}
H_0: R \hat{\beta} = 0
\end{equation}
where $\hat{\beta}$ is the vector of input variables.  For example, if
the variables come as the output of {\em selxavg}, then $\hat{\beta}$
is the set of hemodynamic averages for all conditions.  The number of
columns in $R$ must be equal to the number of elements in
$\hat{\beta}$. The number of rows in $R$ will be determined by the
type of statistic to be computed.  If $R$ has only one row, the the
resulting test must be a t-test.  One can think of the rows of $R$ as
vectors used to correlate with the input variables.\\

While the contrast matrix created by {\em mkcontrast} can be used with
arbitrary data sets, it is geared toward being applied to the output
of {\em selxavg} which produces a hemodynamic average for each
non-null condition, with $N_h$ post-stimulus averages per condition,
where $N_h = \frac{TimeWindow}{TER}$ (TimeWindow and TER are input
variables to selxavg).  If the number of non-null conditions is $N_c$,
then the vector $\hat{\beta}$ can then be conceptually divided into a
string of $N_c$ subvectors each with $N_h$ components.  When invoking
{\em mkcontrast}, the {\em -nconds} options should be set to $N_c$,
the {\em -ndelays} options should be set to $N_h$. {\em mkcontrast}
provides the ability to set up contrasts both between conditions and
between time points within a condition, or a combination of both.\\

\section{Usage}
Typing mkcontrast at the command-line without any options will give the
following message:\\ 

\begin{small}
\begin{verbatim}
USAGE: mkcontrast [-options] -o cmtxmatfile 

Option Set 1:
   -nconds  nconds       : number of conditions (excl fix)
   -wcond  cw1 cw2 ...   : weight of each condition
   -sumconds             : sum conditions
   -ndelays ndelays      : number of delays points in HDR
   -wdelay dw1 dw2 ...   : weight of each delay
   -sumdelays            : sum delays
   -showcmtx             : display contrast matrix
   -sxadat file          : dat file from selxavg

Option Set 2:
   -TER TER              : Estimation resolution (sec)
   -prestim TPreStim     : Estimation prestimulus period (0) (sec)
   -nconds  nconds       : number of conditions (excl fix)
   -wcond  cw1 cw2 ...   : weight of each condition
   -sumconds             : sum conditions
   -ndelays ndelays      : number of delays points in HDR
   -sumdelays            : sum delays
   -ircorr nircorr       : correlate with nircorr ideal HDIRs
   -deltarange dmin dmax ...  : delta range of ideal HDIR
   -taurange   tmin tmax ...  : tau range of ideal HDIR
   -sxadat file          : dat file from selxavg

\end{verbatim}
\end{small}

The two option sets differ in the ways that the weights for
correlating with the average hemodynamic response are set.
Most of the options are common between the two.\\

\section{Command-line Arguments -- Option Sets 1 and 2}

\noindent
{\bf -nconds}: number of conditions (as found in the paradigm file),
excluding the null or fixation condition. See also -sxadat.\\

\noindent
{\bf -wcond}: list of weightings for each condition.  The number in
the list must be equal to {\em nconds}. For example, if there are 3
non-null conditions, then setting {\em -wcond 0 -1 1} would subtract
condition 2 from condition 3 and ignore condition 1.  If one wanted to
subtract the average of conditions 1 and 3 from condition 2, then set 
{\em -wcond -.5 1 -.5}.  The weight can be any real number, positive
or negative.\\

\noindent
{\bf -sumconds}: sum the (weighted) averages of each condition
together {\em before} computing the F-statistic.\\

\noindent
{\bf -ndelays}: number of averages per condition.  This is the same as
the number of components in the time window as set during invocation
of {\em selxavg}.  For a time window of 20 seconds and TR of 2
seconds, there would be $\frac{20}{2} = 10$ averages per condition.
See also -sxadat.\\

\noindent
{\bf -sumdelays}: sum the (weighted) averages of each delay
together {\em before} computing the F-statistic.\\

\noindent
{\bf -showmtx}: display an image of the contrast matrix.\\

\noindent
{\bf -sxadat}: specfy a dat file as created by selxavg.  The dat file
has information about the number of conditions (-nconds), the number
of delays (-ndelays), the TER (-TER), and the prestimulus baseline
duration (-prestim) so that those parameters do not have to be placed
on the command-line.\\

\noindent
{\bf -monly}: only generate the matlab file which would accomplish the
analysis but do not actually execute it.  This is mainly good for
debugging purposes.\\

\section{Command-line Arguments -- Option Set 1}

\noindent
{\bf -wdelay}: list of weights for each average.  The number in the
list must be equal to {\em ndelays}. The weight can be any real
number, positive or negative.  For example, let the number of delays
equal 10. If one wanted to compute the average of the first three
delay averages against the average of the last four delay averages,
then set {\em -wdelay -0.3 -0.3 -0.3 0 0 0 0.25 0.25 0.25 .25}.\\

\section{Command-line Arguments -- Option Set 2}

\noindent
{\bf -TER}: temporal estimation resolution in seconds.  This is the
same as the value supplied to {\em selxavg}.  If a {\em -TER} flag was
not passed to {\em selxavg}, then use the TR. See also -sxadat.\\

\noindent
{\bf -prestim}: prestimulus baseline in seconds.  This is the
same as the value supplied to {\em selxavg}.  If a {\em -prestim} flag was
not passed to {\em selxavg}, then use 0.\\

\noindent
{\bf -ircorr}: specify the number of ideal HemoDynamic Response
Impulse Functions (HDRIF) with which to correlate the average
hemodynamic response.  The ideal response is modeled as a gamma
function of the form
\begin{equation}
h(t) = 
\begin{cases}
0 & t < \Delta \\
(\frac{\tau e^2}{4})
( \frac{t-\Delta}{\tau} )^2 e^{{-(\frac{t-\Delta}{\tau}})} & t > \Delta \\
\end{cases}
\label{hideal.eqn}
\end{equation}
Note that there are two parameters, $Delta$ and $tau$.  Each of the
$nircorr$ functions is given a unique parameter set through the {\em
-deltarange} and {\em -taurange} options.\\

\noindent
{\bf -deltarange}: minimum and maximum ranges of $Delta$.  The values
that $Delta$ assumes will be $nircorr$ values evenly spread through
the minimum and maximum.  Each $Delta$ will be paired with the
corresponding $tau$.\\

\noindent
{\bf -taurange}: minimum and maximum ranges of $tau$.  The values
that $tau$ assumes will be $nircorr$ values evenly spread through
the minimum and maximum. Each $\tau$ will be paired with the
corresponding $\Delta$.\\

\end{document}
